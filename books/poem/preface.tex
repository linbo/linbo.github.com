\chapter{序}

古人云:诗言志,文载道。杜甫也诗云:文章千古事,得失寸心知。似乎作诗填词,写文章,乃阳春白雪,居庙堂之高,凡夫俗子不能,也不可轻易染指的。

虽然从小学就开始背诵“锄禾日当午”,“离离远上草”等等唐诗宋词,但是我不能读懂那些熟悉的文字到底写的是什么意思。我也不懂徐志摩的《再别康桥》,食指的《相信未来》,戴望舒的《雨巷》,舒婷的《致橡树》,海子的《春暖花开》,郑愁予的《错误》,艾青的《雪落在中国的土地上》。但是离别之时,我也会挥一挥衣袖,不带走一片云彩;黯然神伤的那刻,也满怀深情的写着,相信未来;在阴雨连绵的午后,也希望逢着一个丁香一样的姑娘;在春日来临之际,也曾幻想面朝大海,春暖花开;离开江南这么多年了,那达达的马蹄声,仍旧那么地响彻在我的记忆里。

忆着《忆江南》的春来江水绿如蓝,便试了《忆故乡》;读着《沙扬娜拉》的道一声珍重,写了《温柔》;翻着《林教头风雪山神庙》的那雪正下的紧、《塞下曲》的大雪满弓刀,有了《风雪猎》;看着《菩萨蛮》的懒起画峨眉,涂了《伊声》;念着《蜀道难》,作了《醉梦山》;听着《走在雨中》,填了《夜独行》;吟着《雨霖铃》的多情自古伤离别、《别董大》的莫愁前路无知己,挥手《离别》。就这样乱涂乱画,独自娱乐。

言志和千古事估计和我没缘了。也不敢妄自菲薄,说这些是诗词。这些文字,只不过是平常日子的一次心跳,一次感触,一次悸动。

一个声音曾经在耳边回荡:公子便是矫情。不敢矫情,只是用心罢了。
