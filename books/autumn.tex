\chapter{冷清秋}
\thispagestyle{empty}
\renewcommand{\poemtoc}{section}
\poemtitle{冷清秋}
\settowidth{\versewidth}{寒月夜,冷清秋}
\begin{verse}[\versewidth]
寒月夜,冷清秋\\
遥望嫦娥处\\
独舞\\
~\\
酒正酣,故人辞\\
单影对饮时\\
孤醉
\end{verse}

\renewcommand{\poemtoc}{section}
\poemtitle{秋别}
\settowidth{\versewidth}{秋风不知离别味}
\begin{verse}[\versewidth]
秋风不知离别味\\
吹尽落叶为谁别?\\
他年再遇真君子\\
清酒一杯话当年
\end{verse}

\renewcommand{\poemtoc}{section}
\poemtitle{江南秋}
\settowidth{\versewidth}{惜南国之秋无色}
\begin{verse}[\versewidth]
惜南国之秋无色\\
东北望\\
怎奈故园东风无力\\
~\\
可知?\\
江南之秋秋几许
\end{verse}

\renewcommand{\poemtoc}{section}
\poemtitle{秋晨}
\settowidth{\versewidth}{惜南国之秋无色}
\begin{verse}[\versewidth]
清秋的初晨\\
阳光洒尽每一粒空气\\
叶落满地\\
~\\
你把长发飘飘\\
却不肯回眸微笑
\end{verse}
\newpage

\renewcommand{\poemtoc}{section}
\poemtitle{秋天的,最后一片叶}
\settowidth{\versewidth}{无所事事,等待凋零}
\begin{verse}[\versewidth]
我是那\\
秋天的最后一片叶\\
~\\
皮肤已经褪去,青春的记忆\\
肌体也已干枯,没有往昔的活力\\
抱紧枝干,等待凋零\\
~\\
做梦都想,狂风中飘去\\
离开大地的风景,无数次想起\\
看看兄弟,都已离去\\
扑向山川、田野、江河\\
不管哪里,都是大地\\
等待化作春泥\\
~\\
天空阴霾,很是安静\\
鸟儿不再栖上枝头,放声歌唱\\
知了告别多时,从此没有了叹息\\
~\\
风已吹起,没人送行\\
空旷的天空,只有我在飘逸\\
若是离开枝头,不知飘向何地\\
~\\
我已无力,遮风挡雨\\
没有机会倾诉,曾经的足迹\\
但是正如自己,都有过去的风景\\
~\\
我是那,秋天的最后一片叶\\
时刻准备凋零\\
我却不再恐惧,因为还有白雪\\
抚摸着自己,然后\\
永远的睡去
\end{verse}
