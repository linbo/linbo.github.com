早记

夜里开着窗户,一大早,太阳在窗户前来个漂亮转身,毫不客气的把光线踢到我身上,我没理它。

今天阴,有云。

一看8点多了,赶紧起身,洗了个澡,换了套衣服,冲出门上这周的最后一天班。

天阴,没雨。

拉着自行车,在小区门口吃了顿早餐。刚出餐厅,小雨淅淅沥沥,瞬间转大雨。无奈,拉着自行车往回走。

天阴,有雨。

停完自行车,走向公交车站。不远,但打伞。

天阴,雨停。

在公交站台等着,也不知道什么时候雨停了。突然发现道路出奇的干净,各种轿车滚着轮子,积水飞溅,耀武扬威着。

拿下伞,一辆三轮车映入眼帘。此时此刻,这三轮车显得有些另类。高高的压扁的纸箱子堆满了整个三轮车,纸箱下面躺着各种空的饮料瓶。

踏车的男子四十多了,头发不短但不乱,身上的衣服有些破旧了,还沾着一些纸条。他秉着呼吸,目光直视前方,双腿用力的蹬着。他秉着呼吸,努力的蹬着。

当快要到我前面的时候,他终于深深的吸了口气,又慢慢的吐了出来。他下了车,低着头,默默的、用力的推着车。当我再看时,只能看到高高的、压扁的纸箱子在前进了。

这时,太阳出来了,我却看不到一丝阳光


烟

昨夜,娜姐勇夺法网女单冠军,记之。

我点了一根烟,烟夹在手指间,指缝间风儿在荡漾。

烟头不断冒着,点点红光,小小的划破这寂静的夜。

烟嘴冒的烟,如同一条小蛇,慢慢的蠕动躯体,悠悠的向上攀爬,攀爬着,直到躯体也没了影儿。

我盯着这根烟,冒着想法。想法如同烟头的红光一样,扑哧扑哧的跳动着,然后妖媚的化作一缕青烟,消失在空气里。

我感觉有点冷了,于是我熄灭了这根烟。在被窝里,我继续回味着,那些化作灰烬的想法。

问路,是一门学问

今天和同事去洗牙,中间不认识路了。于是一位同事去问路,得到答案,很是欢喜。不过另外一位同事与我都提出异议,认为问路只问一次,只会越问越糊涂。问路的同事听之,认为言之有理,因为他也被问过路,给的答案是错误的。于是又问了一次,果不其然,第一次问的是不对的。

为什么很多时候,问的路都是不对的呢?因为问路是非常非常低回报的行动,是一门需细细讲解的学问。

问路,其实就是一方提供信息,一方根据信息做出响应,然后提供方根据答案进行行动。对完整处理信息的问路过程,就会产生许许多多的问题。

作为信息提供方的问路人。但是由于他对信息掌握不完全,所以他提供的信息是不完整的。这就会给被问路的人产生误解。一般问路人提供的信息都是非常单一的,仅仅是一条路的名字,一个地名,一个商场,一处地标,等等。众所周知,中国的有些路如同长江一样,贯穿整个城市。你要到的地方要是在这种路上,你就要小心了,千万不能只说路名。否则答案说不定和目标有十万八千里。如果一处地标呢?你也得小心,条条大路通罗马,一定要根据具体问题,提供具体信息。本人就犯过这样一个错误。记得08年大雪,偶早早起床去上海南站,坐火车回家。下了地铁,原来的路被封掉了,其它路又没有走过,怎么办?问路呗,问了保安。保安大手一挥,我如同离弦之箭,结果我这只箭如同《新龙门客栈》里的转弯箭,转了一个大弯,最后目送火车离站。假如当时我说明一下我是急着赶火车,说不定结果会好有一点。虽然问路人没有详尽的信息,但是当你问路时,一定要把知道的信息都释放出来。

其实问题出在问路人不多,最多的是被问人了。中国人真是太热情了,一热情,什么事情都可以做出来。有些人明明不认识,突然被你问了,受宠若惊,胡说八道一通。有些人自己都不熟悉,非要当好人,指手画脚一番。还有些知道的,被人家一问,头脑顿时短路,短路出来的有啥好结果啊。所以,哪天你被问路了,没有十足的把握,不要说出非常肯定的答案。

当然有时候人家是很清楚地,可惜他提供的答案实在太简单,或者信息量太少了。结果问路的迷迷糊糊,还没讨教清楚就拍马直奔。结果吃亏得往往是自己。这时候一定要多问几句。

那么既然有这么多问题,怎么解决呢?

首先,问路时候,察言观色。如果别人毫无停顿,脱口而出。说明此人对情况十分熟悉,这种情况成功率比较高。假如那人吞吞吐吐,想了老半天才回答你。或者他说的话自己都不怎么确定,这时候照说谢谢,早点拜拜。

其次,问路时候,找对对象。问路最可靠的对象就是交警这种专门和路打交道的人了。如果路边有这种人,不要犹豫,要觉得自己在马路边捡到一分钱,赶快揣着向警察叔叔奔。如果没有这种人,看看有没有一堆人。如果有一堆人,一定要向一堆人问。即使一堆人提供的初始答案不统一,但是经过内部交流,出来的答案还是比较靠谱的。如果实在不行,千万记得,要多问几次。多一点谢谢,换来少一点弯路,还是非常值得的。

最后,尽量提供多的信息。信息量少了,可能被糊涂处理。信息量多了,处理的答案也可行多了。

问路的,被问路的,千万记得,萍水相逢,几辈子修来的缘分,问路时不妨多聊几句。

我的今年没有春天

也不知道从什么时候起,我开始通过温度来感觉四季了。并非因为我是恒温动物,也并非大自然把四季的温度分得更精确了,而是因为从中学开始,我就被束缚在高高的围墙里。通向大自然的窗口仅剩下四角的天空了。于是我和总理一样,经常抬头望望天空,可是望了老多年,也没有发现四季的痕迹。没有了找妈妈的蝌蚪,没有午休叫个不停的知了,没有火红的枫叶,没有纷纷下坠的雪花。我丢失了打开四季的钥匙,我只能通过一个个死板的数字来了解四季了。

阳春三月了,该是春江水暖的时候了吧。不过通过温度,却怎么也感觉不到春天的到来。现在的鬼天气一时热,一时冷。早上是严冬,下午可就是晚春了。还有这个现在还冷得要死的嘉定,加上过几天就去热带雨林的深圳,导致我的今年没有春天了。

于是我就哼着春天在哪里啊春天在哪里;写着山寺已经芳菲尽,嘉定桃花还没开;读着碧玉妆成一树高,万条垂下绿丝绦;想着天街小雨润如酥,草色遥看近却无;梦着日出江花红胜火,春来江水绿如蓝,以此自我麻醉,忆忆儿时的四季。

不知哪天还可以碰见河边发芽的小杨柳,先知春江水暖的鸭子,破土而出直上云霄的春笋,哭着闹着要找妈妈的蝌蚪。也不知道哪天我还能找到丢失的钥匙,再次打开四季的大门,去看看那记忆中的四季。


吃早餐

实在太无聊了,所以刚才跑出去吃早餐了,为了打发无聊的时间,涂点无聊的东西上去。不知道为什么,耳边想起何勇的《钟鼓楼》:说着明儿早晨是吃油条饼干。

一说起吃早餐,大学生们都笑了。为啥?因为大学生的伙食标准是:早上非洲的,中午亚洲的,晚上欧洲的。逢年过节,亲朋好友来串门,那就是皇室级别的了。我想想原因,可能有如下几条:

一、精力太旺盛,都把时间用在脑力劳动上,比如玩玩游戏之类的,半天没有体力劳动。脑子是饿了,身体却不饿。
二、零食吃的太多。
三、早上懒床,不想起来。

本人就是因为第三条,结果到现在早餐还经常不吃。希望以后工作了,生活有规律了,能提高早餐的标准,解决温饱,奔向小康,一百年不变。

刚才吃早餐,和室友聊到青团。我们都是浙江人,所以都知道我们家乡清明节是要做清明果的。那是用一种植物作为材料的食品,和现在的青团应该有区别的。于是回来路上有思绪飘飘,胡思乱想了。还好是白天,否则恐怕又要失眠了。

以前高中,食堂早餐的类别还是多的,有常见的包子,馒头,油条,清明果等。那时每天都很早起来,估计6点多吧,真是寒窗岁月啊!不过最喜欢的是馒头和清明果。到了大学清明果没有了踪迹,油条馒头偶尔露个脸,堆得小山似的是包子,品种多多,味道一致。用今天的话来说,都是山寨版的。

不知道为什么,我一直很讨厌,极其讨厌吃包子。以至于大学有个阶段,我一看到包子就想吐。但是我现在早餐只是包子+其它。为啥,习惯了。从第一次吃的好吃,到一直吃的厌恶,到现在的麻木,这就是习惯。

习惯是个可敬的东西,当你想改变自己的时候,强迫自己去习惯新的自己。习惯又是一个可怕的东西,因为在不经意之间,在时间慢慢的流逝之间,你已习惯了某种东西,等你蓦然回首,它已经成为你的一部分了。不管怎么说,经意的不经意的习惯,造就了现在的你!


纪念武侠大师-梁羽生

今天才知道梁羽生大师于1月22号已经走了,当年三足鼎立的只剩下金庸了。

第一本武侠小说,记不起来是金庸的《连城诀》,还是梁羽生的《白发魔女传》了。但是梁羽生的作品印象中只看过《白发魔女传》,而一直认为《连城诀》是金庸先生最好的小说。

还依稀记得《白发魔女传》中的白发魔女玉罗刹,武当弟子卓一航,还有白发魔女的干爹?叫啥名字忘记了,武功特别好,好像什么人都怕他,而且白发魔女有时打不过他就出现了。下一代的飞红巾,辛龙子之类的,辛龙子好像吃蛇的。还有白发魔女因为卓一航,一夜之间,青丝全部变成白发。可惜脑海中的具体情节早已烟消云散了。

还记得看武侠时,总幻想着自己也有一身绝顶武功,能飞檐走壁,百步穿杨,宝剑出鞘,一剑封喉;能手中无环,心中有环,刀上无招,心中有招;能在人心叵测,暗流涌动的江湖中打抱不平,惩恶扬善;能在鱼龙混杂,藏龙卧虎的客栈,武林大会中与高手豪饮几十坛女儿红,与前辈高人大战三百回合;能与正派君子,邪教异类日月为证,天地为鉴,不愿同年同月生,但愿同年同月死;能与红颜知己一起闯荡江湖,浪迹天涯。人生如能如此,岂不快哉,还有何求?

金庸,古龙,梁羽生,这三位新派武侠大家,各有千秋。然而一说起金庸,我就想起东邪,西毒,南帝,北丐,中神通。一听到古龙,脑海里出现的就是百晓生的兵器谱。一看到梁羽生,眼前似乎出现炫耀夺目的七剑了。江湖,成了人和兵器的舞台。

金庸全部翻完了,古龙也差不多了,武侠之路还剩下梁羽生的了。有空翻翻,好让自己可以再胡思乱想,痴人再做梦一回。

感谢金庸,古龙,梁羽生,感谢他们在自己最会做梦的时候,给了自己最好的材料,编织了一个个精彩绝伦的美梦!


我有一个梦想

暑假在家,闲来无事,随手翻翻以前的东西。一本染满灰尘的日记簿映入眼帘,轻轻打开,熟悉的名字跳入眼里,原来这是高中毕业时同学写的。看看以前同学的留言,蛮有意思的。最让我惊奇的是,居然有不下三位同学建议我增肥,更有位仁兄居然把我比作甘地,那位绝食的圣雄。可见我当时的身材有多么的苗条。

合上本子,思绪纷飞。自从小学课上回答过“你的梦想”这个问题,似乎梦想再也没有在我脑子里出现了。即使做梦,也没有梦到梦想,原因很简单,我们这代人的路已经铺好了,至少在大学毕业前,你无需考虑你的梦想了。

虽然我素来胸无大志,不过看了这些留言,我发现,其实我还是有梦想的。虽然这个梦想没有惊天地,泣鬼神;虽然这个梦想渺小的简直不能当作梦想;虽然这个梦想多么的不合时宜,但是我仍要自豪的说:我有一个梦想,我要增肥!

要是生长环肥的唐代,我肯定会被人鄙视的一塌糊涂。我要是走在大街上,即使我长袖飘飘,脚步悄悄,那些肥头大耳,身宽体胖的家伙一定视我为丑角,是异类。 我一定羞愧难当,无地自容,即使我长袖再飘飘,我肯定觉得是男儿膝下有黄金,脚下有千斤,以至于我寸步难移,说不定耳边还想起乌龟的嘲笑:瞧那家伙,走路好慢哦!

幸亏我生长燕瘦的今天,所以我走在大街上,都是昂首挺胸,脚步匆匆,旁边不时还飘来羡慕的目光。但是可惜太飘飘然了,没有健步,只有如飞。更让我气愤的是,每次体检时,重如泰山的人上去,那指针来了个加速,然后就停在天文数字上一动不动了。轮到我呢,整个机器哆嗦的不行,那指针在百斤左右的地方来回摇晃,是感谢我轻如鸿毛的身体为他减负呢,还是嘲笑我呢?为此我一直迷惑不解,我不吸烟,不酗酒,我规规矩矩做人,堂堂正正做事,凭什么我的体重只有50公斤,不到!!

虽然偶尔我能跨越一百大关,偶尔欢呼雀跃并不能解开心里的谜团,因为内心深处我知道,这只是偶然事件,不足以证明什么。于是我仔细观察,深入调查,反复检查,试图从现实中找出原因来。即使我花再大的力气,听了再多的建议,做了再多的尝试,仍然于事无补,我还是原来的我。

后来发现,其实弱不禁风的身子也不是一无是处,比如体育考试,立定跳远我每次都轻轻一跃,居然十步,不费吹灰之力得满分。比如挤公交,我每次都游刃有余。而且如果为了单纯的增肥,胡乱饮食,结果让身体受到损害,我岂不是本末倒置了吗。

恍然大悟:其实只要生活的有规律,饮食健康合理,管它燕瘦环肥呢!于是,我又成了没有梦想的人了,我又要寻寻觅觅了。


没有阳光的房间

没有清泉的地方是没有灵性的,没有阳光的地方是没有温暖的。

学生时代尤爱散文,最喜经典文章的片段,闲来无事,也会情不自禁的低语几句。比如朱自清先生的《匆匆》。

每当低语轻诵太阳他有脚啊,轻轻悄悄地挪移了...我便憧憬有一个充满阳光的房间。每天早上,阳光毫不吝啬的,大片大片的洒进我的房间。我可以懒洋洋地躺在床上,默默的感受太阳它那轻轻地脚步。

可惜事与愿违,研究生选的是阴面的宿舍。晴天一拉窗帘,外面艳阳高照。但是只能目视太阳挪移的脚步,却从来没有在房间里享受过。可望不可即的阳光,让我知道了什么是一失足成千古恨,悔不该啊!

既然选择了,就要勇于承担。在第一个学期,我抱着一个热水袋,硬是熬过了一个冬天。为此,在学期期末复习时,曾经写了一首打油诗:

寝室如两极,图书馆似赤道,偌大嘉定村,何处安我身?

冬天走了,下一个冬天还会远吗?自从“台风”吹来一个秋天后,我就惴惴不安,恐怕哪个“韩流”会带来冬天。那个陪我过完一个冬天的热水袋,因为经受不住我 的摧残,已经告老还乡,不能为我服务了。秋天象征性的露了个面,就消失的无影无踪了。冬天终于张牙舞爪的来了。刚开始,我还能抵挡一阵子,但是没过多久, 当我坐在房间里时,周围弥漫着阴冷,它源源不断地向你涌来。这些寒冷走的是农村包围城市的路线,有外而内,不知不觉穿过你的衣物,透过你的肌肤,侵入你的体内。当你意识到寒冷时,你已经全身心的淹没在寒冷之中了,你只能缴械投降,要不奔向赤道,要不躲进同样冰冷的被窝。

阳春三月,又回到嘉定,又回到没有阳光的房间。天真的想,阳春三月啊,房间应该有点温度了吧。但是,想法是美好的,现实是残酷的。头一天坐在房间里,阴魂不散的寒气又来了。雪上加霜的是,今天又下了一场贵如油的春雨。我只好跑到赤道,于是我完成了在赤道的第一篇博客。

买房的,找房的,一定要记得我的金玉良言:今年如果要找房,一定要找有阳光的房啊!


知道我的兴趣吗?

  我到底对什么有兴趣呢?我一直在想这个问题,但是没有答案。也许答案一直在风中飘着呢。
  于是写写文字,说不定可以一窥我的兴趣这个冰山的一角。
  知道“红门痞女”洪晃不?不知道吧。但是你一定知道她的前夫,大名鼎鼎的导演陈凯歌。如果你还是孤陋寡闻,那你一定知道“一个馒头引发的血案”吧,还不知道?你不用看下面的文字了,我们不是平行线就是空间上的异面直线,永远也不可能相交了。
  知道的继续了,洪晃的母亲就是号称“中国最后一位正牌名嫒”章含之(也不知道谁说的)。洪晃的继父是笑傲国际外交战场的乔冠华,可是中国的外交天才虽然可以纵横国外的外交战场,但是对国内战场却一塌糊涂,真是可惜。
  说完父辈说祖辈了,这里故事太多了。洪晃的祖父就是著名的爱国民主人士章士钊。注意不是写《往事并不如烟》、《伶人往事》的章诒和的父亲,章诒和的父亲“中国头号大右派”章伯钧,而且至今未得平反。都是著名的爱国民主人士,但是两个人遭遇天差地别,令人感慨万分。
  章士钊估计很少人知道,但是我一说一篇文章大家就知道了。鲁迅先生曾经写了一篇《纪念刘和珍君》的文章。当时北师大闹学潮,要罢免当时的校长杨荫榆女士,去段祺瑞执政府前请愿,结果酿成惨剧。当时的教育总长就是章士钊,结果“老虎总长”章士钊被迫辞职。
  《纪念刘和珍君》的故事太多了,如果有兴趣可以去查阅《读书》还是《随笔》的一篇文章,记不清了。会发现段祺瑞真的是那样滥杀无辜吗?国民党在其中扮演什么角色?许广平当时为什么没有一起去请愿?请注意当时历史背景。
  不过《纪念刘和珍君》真的是好文章。
  再说说北师大,别小看北师大这个称号。大学可以罢免校长,中学做出的事情更让人目瞪口呆了。文革中,全国第一个被活活打死的老师就是北京师大女附中第一副校长(没设校长)卞仲耘。看看中国的近现代史,特别是五四运动,以及上世纪八十年代末的事件,就会发现北京高校的革命精神是走在全国前面的。
  最后是杨荫榆女士,由于鲁迅先生的一篇文章,杨荫榆女士似乎成了反动军阀的帮凶,封建余孽的化身。但是在日本侵华期间,她的表现可比美男子汪精卫强多了,最终惨死在日寇的屠刀下。
  想了解杨荫榆女士的可以看看杨绛先生写的《回忆我的姑母杨荫榆》。杨荫榆是杨绛的三姑母。杨绛先生是谁呢?杨绛先生作品有《干校六记》,《洗澡》,《我们仨》等,是非常有才华的一位女作家。说到杨绛,不能不说说杨绛的丈夫,大名鼎鼎的...。如果不知道,肯定知道《围城》吧。可是这位才高八斗,学富五车的国学大师已经离开我们好多年了。
  流水帐记到此为止,哪天想到别的,再记记。
  有空时,停下脚步,观察自己,认识自己,还是很有必要的。


80后,动画片

  80后的童年都是用动画片孵出来的,要不,怎么一部《变形金刚》,把80后的童年记忆集体找回来了呢?但是如果我严肃地告诉你:有个80后的家伙,童年没有经过动画片孵化,就出来了,你信吗?你肯定不信,而且我能猜出你接下去的行动:首先,看这家伙的身份证,是不是80后的。如果是,接着看看这家伙智商是否正常。如果是,然后,找个测谎仪看看这家伙是不是说谎。如果不是,最后,老虎凳,辣椒水,竹签,火盆,美人计, ....看看这个家伙是不是混进80后的奸细。
  但是,我十分肯定的告诉你:我相信。因为这个家伙就是我!我84年出生,身份证上写得清楚明白。我在读研究生,智商还算正常吧。测谎仪,严刑拷打,美人计就不用了。我如果说自己童年看过那些经典的动画片,那才是睁着眼睛说瞎话呢。即使山无棱,天地合,我的童年也不会和动画片有瓜葛。
  要说我的童年和动画片一点点关系都没有,那是不可能的。俗话说:没吃过猪肉,还没见过猪跑啊。不过要与时俱进,用发展的眼光看待这个问题。如果像今天这样,猪肉价格飞涨,猪都成奥运猪的时候,你恐怕不但吃不到猪肉,还真没有见过猪跑呢。照这样发展下去,哪天唐僧师徒四人同游南京路,满大街的外星人和地球人估计就会出现这样的对话
  外星人(指着猪八戒):看,那是什么?
  地球人:地球人都知道啊,熊猫头,告诉你吧,这是地球宝,猪啊。
  不过猪八戒也别得意太早,因为早有人在暗处磨刀霍霍,只因为一个遥远的传说:吃了猪肉可以长生不老,据说唐僧就是因为出家之前吃了猪肉才长生不老的。
  在我印象中,我小时候有一天晚上看了《恐龙特急克塞号》,除了这个名字,我还真记不起其它东西呢。不过就凭这一点,我也算是80后一分子了。我不但见过猪跑,我还吃过猪油呢。
  其实,我和动画片的缘分并不是这样简单的,谁叫我是80后呢。终于,我不但吃过猪油,我开始品尝整只猪了。记得是在高三,正是高考复习热火朝天的时候,我们学校很早就实现小康,每个班级有台电视。正巧那时在播放《灌篮高手》,呵呵,如果以动画片来判断是否是80后的话,我看着90后的动画片,就这样成为真正的80后了。
  年年岁岁花相似 岁岁年年人不同。不过毕竟年代不同了,看动画片的各个方面也不一样了,所以也不能体会到童年看动画片的感觉了。记得当时感觉这动画片有点拖沓冗长,那个木木投一个压哨三分,好像放了三集。真不愧是木木啊。
  以为自己和动画片的缘分已尽,殊不知,大学我又和动画片邂逅了。应该在大二,居然看了一遍《圣斗士》,忘记哪一篇了,反正是雅典娜和黄金圣斗士基本上全下地狱的那部。看《灌篮高手》多少因为篮球,看《圣斗士》多少因为星座,西方诸神等等。能从动画片中知道一些其它东西,也不枉动画片是80后的标志了。
  作为80后的异类,虽然没有动画片的童年,但是还是拥有了动画片。于是也可以骄傲的对世界说:我也是80后,我也看过动画片!



游戏盲的悲哀

  "嘿,哥们,今天玩什么游戏呢?"
  这是很多人给我的见面礼。每当我收到如此丰厚的见面礼时,我总是无地自容,羞愧万分,因为我是一个不折不扣的游戏盲。
  其实我本来可以成为众多游戏爱好者中的一员。本科时候,寝室有不少兄弟奋战在游戏战场上,兢兢业业,无怨无悔。我也尝试了几次,但是都无果而终。我边上的玩拳皇,天天巴神。某天我也小试了一把,但是我老是记不住那些键,最后我对那些键缴械投降,失去了信心。我上面的玩魔兽,对面的玩三国(统一了n次,乐此不彼)。不过我一直存在一个问题,为什么游戏中砍人都是很公平的。我砍你三刀,你还我三刀,决不会多出一刀,也不会少一刀。连至尊宝都可以让别人还三刀,在游戏里居然不可以。在这弱肉强食的社会里,居然有这等公平和谐的事情出现,我还真接受不了,所以我就成了游戏盲了。
  我曾经因为自己是游戏盲而自豪,以为举世浑浊而我独清,众人皆醉而我独醒,天天飘飘然的生活着。要不是有地球引力,说不定飘到太空去了呢。但是,某天,我被现实从万丈高空拉了下来,直接自由落体到万丈深渊,差点就永世不得翻身。事情是这样的,和朋友出去玩。当然,根据朋友关系的传递性,就有朋友的朋友。既然有朋友的朋友,那就是新朋友。既然是新朋友,那就得寒暄几句。既然要寒暄几句,肯定是我们共同的话题。那哥们张口便问:"同学,你玩什么游戏。"条件反射:"我不玩游戏。"在那灿烂的阳光下,我好像成了阳光下的罪恶。可怜那家伙张着惊讶的大嘴,半天没有合拢,好像在说:嘿,哥们,你不是吧,连外星人现在都玩星际争霸了。我终于明白现实的残酷,因为即使我飘到太空,遇到一个外星人,那家伙一高兴,拉我去星际争霸,个人失节是小,给地球人丢脸可是大问题啦。想到这些,我手足无措,大汗淋漓,半天没有缓过神来。
  后来我又发现一个更为严重的问题,让我对自己是游戏盲更加悲哀了。玩游戏所让人诟病的是浪费时间,我也常常因为没有把时间浪费在游戏上沾沾自喜。但是有一天我看到一段话,让我因为没有玩游戏而后悔不已。不抽烟的朋友指责抽烟的朋友:"你瞧你,天天抽烟,要是不抽烟,你抽烟的钱可以盖一座房子了。"抽烟的说:"丫的,怎么没见你盖起新房子啊。"我发现,虽然我没有玩游戏,可是我把玩游戏的时间浪费在更无聊的事情上,而且这些事情没有给我带来玩游戏的快乐,也没有给我带来玩游戏的记忆,更不会像玩游戏一样交上不少朋友了。甚至,甚至连练习键盘的机会都没有。
  于是我顿足捶胸,向天发誓,我也要玩游戏。正当我准备笨鸟先飞,奋起直追之时,我考上了研究生。我那个高兴啊,简直难以用言语表达。因为我以为我可以在这里找到知己了,我再也不会收到那见面礼了。某天,新同学见面,第一句:"同学,你玩什么游戏啊?"我彻底崩溃了...
  不过到今天我还是游戏盲,而且我再也不会因为游戏盲而悲哀了,因为现在我收到一份更大的见面礼,那就是:
  "嘿,哥们,今天XX股票涨了没有?"


我的手机掉到水里了

  去年暑假的一天,高龄的不能再高龄的太阳公公,吐着毒辣的不能再毒辣的舌头,发着巨大的不能再巨大的能量。万物耷拉着脑袋,无精打采,连水泥地板也散着白气,奄奄一息。我无所事事,梦游般地在屋里游荡着。这时,一件事情发生了….
  我的手机掉到水里了!
  说时迟那时快,当我反应过来时,被我当作时间来用的手机已经舒舒服服的躺在水里,嘴里还冒着小气泡,似乎它也消受不了太阳公公的热情,于是不顾一切的奔向阴凉的水中。
  来不及多想,我马上捞起时间,拿到太阳底下。
 v然后,我就在想一个问题:我为什么没有在手机落水前,阻止它的自杀行为。根据物理的自由落体知识:h=1/2*g*t*t。当时手机的高度大概1m,g就算10吧,我算了算,下落时间是不到半秒。想到这个结果,我也就没有什么内疚了。
  后来我还是内疚了,因为我把手机放到太阳底下了。
  自从唐僧说下雨了收衣服了以来,人们都知道,东西要晒晒才会干。这可是亘古不变,天经地义的真理,可正是这个真理,使我内疚了。
  很久我就注意到这么一个情况,那就是本来干的衣服,被一阵暴雨淋了几分钟,就湿透了。然后在艳阳高照的晴天,可能要晒一天才能把几分钟的雨水从衣服中蒸发出来。所以从某种意义上说,用太阳晒东西是下下策。
  当我醒悟时,我发现我的手机居然还可以工作,其敬业精神为之动容。为了减少我心中的愧疚之情,我决定让我的手机休息休息,于是我给它放了两分钟的假。
  然后,我的手机罢工了。
  此时的我,如同热窝上的蚂蚁,但是即使我顿足捶胸,哭天喊地,声泪俱下也无济于事。后来我做了好多好多实际的工作,手机才同意开工,不过已经元气大伤,时不时闹罢工,而且那节电池也落井下石,不但闹着闹着要加工资,还把自己的工作日从三天缩短到半天。
  后来听说,当手机掉到水里时,马上捞起来,放到---冰箱里!
 v有时候做事情,全凭一些习惯,就好像惯性一样,根本是不用大脑的。所以感觉人骨子里还是感性的,只不过添加了一层理性的外衣罢了。为了纪念人的感性,所以写了这篇文章。当然,还不忘替诺基亚做一下广告,因为我的手机是诺基亚,大难不死之后,还得为我服务。


冬夜小事

  夜已深,极黑。
  斑驳树影,惨淡月光,稀疏人影。
  参天树木下的小径静悄悄,偎依在树木旁的路灯正洒着桔黄色的,温暖的灯光。
  风大,自习归来的我把自己裹在衣堆里,发着抖,低着头,飞奔着。
  寒风呼呼,脚步匆匆。
  突然,前面一个巨大的黑影映入眼帘。一抬头,吓一跳,一个高高的身影已向自己悄然接近,躲在黑暗的帽子底下嘴正吐出白气,被风吹起的黑衣在风中张牙舞爪,甚是可怕。
  心砰砰乱跳,两只不大听使唤的脚机械的向前迈着。
  正当我们要擦肩而过时,忽然发现那戴着黑色手套的左手以极快的动作伸向衣兜。
  正当全部神经都要伸懒腰时,突然,寒光一闪,只觉的自己脖子一凉,继而一热,提到嗓子眼的心如同断了线的风筝,扑向无底的深渊……..
  几秒钟后,发现没有了重力加速度的心还在跳,无知的脚还在机械的往前迈。才飘出一口气,也是白的。
  等心做匀速运动,神经都昏昏欲睡时,细细一想,才发觉刚才那个人拿出的是手机(一种也可以在黑暗中发出寒光的高科技产品)。
  至今想不通为什么我当时没有听到手机铃声!



但愿这是最后一次———–看昨晚中青比赛有感

  当孙继海还是国奥队员时,就开始关注中国足球了。如今孙继海们逐渐老去,郑智们正值当打之年,孙祥们崭露头角,到现在的国青,算起来已经有四代球员了。在这几年的时间里,中国足球带给我的已经不能用几个词来形容了。
  当开始关注中国足球起,我丝毫不知自己已经进入了十八层地狱,而且竟默默在痛苦中煎熬到现在。每当国字号比赛时,我都陷入万劫不复的深渊。每次比赛结果都让我那么的心碎,那么的欲哭无泪。曾经无数次仰天发誓,今生今世再也不看国字号的球了。然而希望就如同野火烧不尽,春风吹又生的野草一样,我一次又一次的出尔反尔。为什么我如此的言而无信,因为我对中国足球爱的深沉。于是乎,希望,失望,绝望…..
  终于痛下决心,对中国足球说,永别了!当同学异常兴奋地告诉我中青的精彩表现时,我不禁暗自窃笑:看吧,希望的后面就是绝望了,连失望都不会给你的。后来禁不住看了与乌克兰的比赛,我知道,这次我也许错了。
  相对于永远都在梦游的后防线,找不到东南西北的中场,眼里永远没有球门的前锋线的国家队,国青真的是天壤之别。
  为什么希望又有呢?最主要的就是国青的精神面貌。进球的国家队如同缩头乌龟,失球的国家队如同憋气的气球。但是国青不同,进球了,还得进球,失球了,还是要进球。在他们身上看不到中国特色,这是中国足球最最缺乏的。第二就是技战术水平。国家队的比赛,传球的不知道东南西北,接球的不知道前后左右,而且整场比赛看不到一次过人,前锋射出的球不是对守门员的投怀送抱,就是对看台的球迷来个亲密接触。但是国青们敢突,敢过,敢带。精准的传球,精妙的配合,惊艳的盘带过人。远程轰炮,凌空抽射,带有美妙弧线的任意球。以至于我看比赛时说,这是欧洲联赛水平的。虽然我不敢怎么夸大国青的技术,但是,国青们与世青赛中任何一支队伍比赛,至少在技术上是不会处于下风的。
  于是乎,中国足球是不是有希望了,是不是崛起了,是不是该先灭韩日,后灭阿巴呢?非也,非也。曾经的超白金一代,经过中国大地的滋润,已经慢慢失去了光泽。幸运进国家队大染缸的,已经被精练的只剩下一堆废铜烂铁。况且一堆不懂足球的县太爷们在那些指手画脚的指挥着。哈哈,终于知道桔生淮南淮北为什么不同了。
  如果几年之后这些球员也如同现在国家队的那些一样,我只好对足协的县太爷们说,求求你们,散伙吧,别把我们曾经无数次失望的,仅有一丁点的希望也给残忍的扼杀啦。
  我幼小的心灵再也禁受不住摧残,我伤痕累累的心灵再也是不能添加新的伤痕了。
  我希望这是最后一次了。


娃的哭声

昨天,同事跟我讲了这么一个小事,Ta说Ta家不到一岁的娃太胆小了,被大一点的姐姐摸了一下,哭了。奇怪,大哥哥我摸他的时候,他还笑着呢。

经过昨天死睡死睡之后,今天精神好多了。早上快到公司的时候,我深深的吸了吸那洋溢着阳光的空气,脑子突然跑的比CPU还快,于是我又好好的思考了娃的哭声。

对于同事的结论:娃胆子太小了。我是持否定态度的,不过当时给了一个解释,没有具体的分析过程。思考一番,觉得有两个问题需要解答

小孩为什么哭

不到一岁的小孩,在被陌生人摸了一下,哇哇哭了。抛开前提,不到一岁的小孩哭了,我觉得比较多的可能是饿了,或者更准确说不舒服了。比如说尿布湿透了不舒服,被摸了一下不舒服,肚子饿了不舒服,抱着的姿势不对不舒服,这些小孩都会哭。所以陌生人摸了一下不一定是小孩哭的原因

小孩胆小会哭吗

小孩哭了,就表示胆小,我觉得这个也不是很严密。胆小的小孩不一定会哭,况且我觉得不到一岁的小孩,应该还没有胆小的概念吧。

所以,仅因为娃娃被小姐姐摸了一下哭了,得出娃胆小的结论,其实是有很大漏洞的。但是问题还没完了,你们这些为人父母们,让我来问一个问题

为什么得出这个结论?

这个问题我也没有答案,我写这篇小文章,大部分是为了让看的人去思考这个问题的。我现在思考的答案是:

自身经验
我们都是从小孩到大人的,也许自己小时候胆小,爱哭,或者看到其它小孩胆小,爱哭,所以就有这样一个观念,小孩碰到代表胆小的事情哭了,说明小孩胆小。其实细心想想,我们现在的很多观念,都是胆小的观念。不然你自己细心体会一下,逻辑分析一下,你会发现漏洞百出,而你一直把它当作真理。不信?看看下面一句话:

婚姻是终身大事

如果这样的话,你的终身不包括呱呱落地,呀呀学语,书声琅琅了? 当然这句话也可能仅仅是做了艺术性的夸张,我也不反对婚姻的重要性,只是这样称呼觉得太露骨了一点点。

外界经验
我一直很奇怪,比如一些东西传播的很快。比如神药板蓝根,包治百病,被疯狂购买n多回。我思考了一下,觉得是因为恐惧,或者说对未知的一种反应。换句话说,对未知的信息,你更容易相信。比如你让农民相信亩产三万斤看看?但是你让不食人间烟火的知识分子,精力无处发泄的红卫兵相信,估计有些人会相信的。

所以人们更愿意相信权威,而外界经验很多都是权威慷慨分享的,所以许多人对外界经验视作真理。我总觉得,权威的话固然重要,也可以学习吸收,但是不加消耗全盘吸收,老怕营养过剩。

不管自身经验也好,外界经验也罢,总之,要经得起?的考验

后果

对于这个的后果,我是这样想的。由于父母对小孩产生胆小的误导心理,所以以后肯定会多加注意,势必有意无意的把小孩当成胆小的小孩来对待。如果小孩得到这样的教育熏陶下,不知道会不会潜移默化的影响小孩,让他觉得自己真是个胆小的小孩。这个我有点担忧。

后记

虽然不懂心理学,教育学,而且可能全篇都在漏洞百出的胡说八道,不过我觉得至少我思考过了。如果你有更好的想法,分享分享
