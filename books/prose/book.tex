\chapter{书意点滴}
\section*{一个傻子的故事}
\addcontentsline{toc}{section}{一个傻子的故事}
\markright{一个傻子的故事}

缘由

虽然无聊的时候我喜欢哼哼几句,涂抹几笔,但是我是绝不敢写影评,读后感之类的文章的。一方面看了所谓影评家、文艺评论家的文章,实在不敢苟同,总觉得他们在自个儿玩味自己的感觉;另一方面,其实很多电影,小说我是看不懂的,我根本不知道他们要说啥玩意,所以我不敢写,因为我不知道写什么东西。关于第二点,我觉得中国教育立了大功,小时候老是被逼着写调子一样的读后感,搞到现在一看不懂文章,二想写读后感都是一样的思维,三没啥想法了。

但是翻了一些书,看了一些电影,总有些思想的青烟在湛蓝的天空飘着,飘着,然后慢慢的淡化着,淡化着,消失了,不曾留下半点痕迹。

于是,为了让自己有点想法,为了那些青烟可以变成朵朵白云,化成春雨,滋润自己,我还是想点什么记着吧。

最近看了阿来的《尘埃落定》,没明白写什么,但是又有什么关系呢,先胡扯几句再说呗。

傻子

在麦其土司辖地里,没有人不知道土司第二个女人所生的儿子是一个傻子
那个傻子就是我

我欣赏聪明人,但喜欢特别的人。傻子不笨,傻子是特别的人,我喜欢傻子。

傻子是麦其土司酒醉后和汉族的女人生的,大家都知道他是个傻子,包括他自己。傻子知道自己是个傻子,真他妈的是个傻子。

傻子傻,我看完后,没有印象了。

傻子傻,傻子不笨,因为傻子喜欢女人,傻子聪明的哥哥也喜欢女人,傻子智慧的土司父亲也喜欢女人。傻子受不了诱惑,中了茸贡土司的美人计,因为那是天底下最美的女人。傻子的哥哥受不了诱惑,因为那是天底下最美的女人,傻子的父亲受不了诱惑,因为那是一个漂亮的女人。傻子的哥哥受诱惑时,他嫉妒着。傻子的父亲受诱惑时,整个土司弥漫着罂粟的味道。傻子受诱惑时,眼前站着的是天底下最美的女人。

傻子傻,傻子不笨,傻子有智慧。说傻子有智慧,估计会侮辱聪明人了。那就按聪明人看傻子的眼光来说,叫傻子能看得到未来。傻子能看得到的未来,是聪明人看不到的,而智慧的人看得到。一个傻子如果能看得到未来,他在聪明人眼里就不是傻子了,他在智慧人眼里,就不是傻子了,他成了一个让人担心,让人害怕的傻子。但他终究只是一个有智慧的傻子。

宗教

土司辖区有自己的宗教,有喇嘛,有活佛。土司需要宗教,宗教需要土司,仅此而已。

上帝的代言人查尔斯来到这里传教,第一场雪后,他走了。

圣城拉萨的翁波意西来这里圆梦,没有走,结果就是他被割了舌头。他还是没有走,只不过他成了奴隶,成了书记官。他还是没有走,他可以说话了,他一直是写的,但是那一次他说话了,然后他又一次被割掉了舌头。

土司里面不会有新的宗教。

汉人

麦其土司被欺负了,没有找宗教,但是去找皇帝了,因为土司的职位是皇帝封的。民国了,没有皇帝了,政府派了代表来了,一个汉人,一个带着先进武器的汉人,一个带着罂粟种子的汉人。

麦其土司赢得了战争,土司辖地里种满了罂粟,土司仓库里堆满了银子。于是所有土司都去种满了罂粟。

麦其土司和傻儿子一样,看见了未来。于是傻儿子成了富甲一方的人,成了土司中的土司。然后,汉人来了,有颜色的汉人来了,谁也阻挡不了。

土司里面不能容忍新的宗教,但是土司抵挡不了有颜色的汉人。

傻子跟母亲说:阿妈,有颜色的汉人来了。汉人阿妈终于没有说什么。

尘埃落定

傻子看见了未来,但却不能有未来,因为他是最后的土司,土司中的土司。他注定要和所有土司一样,被颜色吞没,消失的无影无踪。

但是,如果有来生,他一定还是那个傻子,一定还在那个美丽的地方。

掩卷

如果人生可以有多种选择,那么,成为聪明人群里的傻子,一定是我的一个选择



\newpage
\section*{江城,那消失的过去}
\addcontentsline{toc}{section}{江城,那消失的过去}
\markright{江城,那消失的过去}


没有人可以像 彼得·海斯勒 那样,如此细致的观察,描绘,思考着现代的中国

《江城》里的文字太熟悉了,里面的一字一句就是我十几年前的生活。翻着文字,回忆便如汹涌不尽的长江水,扑面而来,那般亲切,那般浓烈,那般无情。而我,如同长江上面的穿空巨石,突兀挺立,不断被记忆之水撞击着,碰撞出无数的水珠,在空中飞溅着。我手忙脚乱的拿出手机,记下那些撒向空中的水珠。我再也不敢看下去了

但是,那些日子,已是我的一部分了,再也无法分开了,我能做的,也许就是在微风徐徐的午后,雨后夕阳的黄昏,夜深人静的晚上,再看一段,再忆一次,再温暖一回。

作者与作品

彼得·海斯勒是美国人,在1996年至1998年以“和平队”志愿者身份在四川的一个小县城 - 涪陵当外教,并陆陆续续在中国生活了十年。在十年期间,彼得·海斯勒完成了中国纪实三部曲《江城》,《甲骨文》,《寻路中国》。我觉得,不管你爱不爱看书,不管你关心不关心现代中国,不管你是否生活在中国,你都应该花点时间读一读。

当然这三部曲都是英文的,不过现在都已经有中文版了。我最先看的是《寻路中国》,然后是《甲骨文》,最后是最近才看完的《江城》。看完《寻路中国》与《甲骨文》,似乎中国现在的种种,被彼得·海斯勒投影到纸里,竟然如此般鲜活。而《江城》,便是十年前的中国的投影了。

我偶尔也痴人发梦,想写一些文字记录我所走过的日子,因为作为承上启下的八零后,每一段时光都见证了现代中国的变迁。我把自己过去的时光分成几个阶段,第一个阶段是出生到小学毕业,那时我还在浙西的一个大山里。第二个阶段是六年的中学,那是我走出大山,走向外面的第一站。第三个阶段是大学研究生,那时我已经走出曾经养育我的土地,投入到现代的中国,而且永远也回不去了。

旧年我看路遥的《平凡的世界》时,已经发现原来作品可以把一个特定的历史时期永远的定格,就如同相机可以把一切都可以定格一样。当你再回过头,翻着这些作品,看着这些老照片,曾经记忆里时光如同电影一样,在你脑海里一帧一帧的流动着。你会发现,原来这些时光我也走过。《江城》,就是我第二个时期的老照片,就是我过去时光的灿烂影像。

那消失的时光

1996年还是1997年,我去家乡县城上了初中,我并不知道那时有个老外跑到千里之外的四川小县城当外教,然后在差不多快20年之后,我和他的过去在《江城》里相遇。

彼得·海斯勒忠实着记录着他在涪陵2年的时光,在2年的时光里,他参加了各种各样的活动,有纪念学校长征的集会,然后稀里糊涂的被拉上主席台,被领导称之为同志的荒诞经历;有参加马拉松比赛轻易夺取冠军,结果却让他黯然神伤,并不再参加任何比赛;有各种各样的领导欢迎他们的宴会,并和领导拼酒,努力理解酒桌的文化;和大家一样,哀悼邓小平的逝世,旁观着香港的回归。

当然,作者在2年多的时间了,积极体验着那时的中国。于是,他被小姐纠缠过,被脑子不大好的大龄女青年追求过,和小摊小贩大吵并差点兵戎相见,莫名其妙的参加了葬礼,站了48个小时的火车(从新疆到成都),也乘船只溯江而上,顺将而下,体验着“两岸猿声啼不住,轻舟已过万重山”。

不过让我惊奇的是,四川最贫穷的县城之一涪陵,似乎并没有那么落后。那时市长坐着的是奥迪,那时很多人在炒股,而且居然有外教。我家乡号称浙江的西藏,我们学校有外教好像也是我读高一的时候了。

看着这些文字,我只能断断续续的忆起我的初中高中的生活,共鸣之处,会心一笑,然而我终究把逝去的日子抹在了记忆的沙漠里,似乎只有偶尔来袭的暴风,吹走流沙,那记忆的遗址才能显山露水,供我眺望了。

那消失的思考

回望着过去,我觉得中学时代,是我思想最受禁锢的时期之一,每天的任务就是读书,而从来不会自由的思考。看着《江城》,你会发现,原来单调乏味的日子,原来并非如此,只不过自己没有用心思考罢了。而彼得·海斯勒却不断的思考,并付诸文字,见证着生活的厚实。

彼得·海斯勒的三部曲推荐之原因,就是书里面处处体现的自由之思考,可是我无法将这些思考当作书评写下来。能做的,只能希望你少看几集电视剧,少逛几次街,少贪睡几小时,用一点点时间,看看这些文字。

以前一直会有这种感觉,过去的几个月,几个星期,却颓然发觉好像都是一个样,过去的时光是一个模子里面的。于是哀怨时光无价值流逝,并惶惶而不知所终,其实最后发现,只不过是没有像彼得·海斯勒那样,学着热爱涪陵,学着思考涪陵,学着记录涪陵,学着回忆涪陵。

我想,当哪天你从心底热爱着你的生活,并不断思考着,生活也许就不是那么单调乏味,而是厚实了。也许,那一天,你会有一本自己的《江城》。


\newpage
\section*{读史读政治}
\addcontentsline{toc}{section}{读史读政治}
\markright{读史读政治}


《中国历代政治得失》是大师钱穆的中国政治制度史专题演讲合集, 很薄,才160页,我特么感觉很多书是那么的厚颜无耻。16块大洋,真是暴殄天物。为此,我认认真真的看了2周多的时间。

大师演讲了五个朝代(汉、唐、宋、明、清)的政治制度,分别从政府组织(中央政府、地方政府)、选举制度、经济制度、兵役制度四个大方面,简明扼要的阐述了中国具有代表性朝代的政治制度。虽然政治制度不是每一个人都感兴趣,但是这书至少在几个方面,值得翻看。

历史知识

中国历史实在久远的可以,所以电视剧里的宫廷剧简直泛滥成灾,于是各种历史知识之欠缺,简直就是罄竹难书。如果那些编辑看过此书,也就不会闹那么多笑话了。

书里面讲了很多历史知识,比如在汉代里面,讲了三公、九卿各自指什么,具体负责的职位是什么,是由前面朝代的哪些职位演化过来的。比如位列三公的丞相是最高的行政长官,而太尉就是武官首长了。像这类的知识性介绍比比皆是。

除了知识性的介绍,还对一些历史知识,抽丝剥茧,追根究底,娓娓道来,看完之后,你才发现,原来如此。比如丞相为什么又称宰相呢?原来

一、在封建时代,贵族家庭最重要的事是祭祀,而祭祀最重要的是宰杀牲牛,于是当时替天子诸侯等贵族共卿管家的都称宰。
二、在封建时代,在内管家称宰,出外副官称相,所以宰相到汉代,宰相即要管国家政务,还要管皇帝的家务

还有就是对政府组织结构变化之描述,比如汉代是宰相,到唐代变成了中书、门下、尚书三省了,宋代就一个中书省了,明代成了六部尚书,清代则是军机处等等。通过政府组织结构之变化,可以看出历朝历代之政治面貌,也就会知道宋代为什么那么弱,明清两代皇帝为什么更加独裁。

历史形态

以前说起中国的各朝各代,实在薄陋的可以,其实想想,中国那么大,如果历朝里的就只剩下皇帝、后宫、朝臣,支撑得住这个国家吗?所以说,看了本书之后,历朝历代之形态已不是细柳扶风,而是体态丰满了。

书中讲述了很多关系,比如皇权与相权之对立发展之关系,地方政府与中央政府的关系(我们现在的地方政府与中央政府,不提也罢),军队制度与朝代更替之关系,历朝历代税收如何影响底层的农民。总之,看完此书之后,你会发现历史存在感突然充实了起来,也会发现今天之中国,似乎某些部分也在历史中有迹可寻。

历史态度

钱穆大师一定对当时很多人抨击中国朝代之专制黑暗极其不满,于是才会洋洋洒洒了连续了五次2小时的演讲。在前言中,也提出“历史意见”与“时代意见”之说。通篇不时驳斥不研究历史,而一棒子打死历史之观点,并不时提出自己的观点:各种制度有其存在之原因,不能因为其某一方面之缺陷,而全盘否定;任何一种制度之出现,都有时代之原因,而任何好的制度,如果不因时而变,过了两三百年,也会成为坏制度,阻碍发展,但不能因为成为坏制度,而不见其出现时之优点。

对待这种对待历史之态度,很值得学习。我们对太多事情有太简单的看法,最悲催的是,这些看法还不是你自己的,是你从别人那里偷来的,没错,就是偷来的,你压根儿没仔细推敲,辩证过这些想法,并理所当然的认为是正确的。反对这种看法的,就是xxx,真是悲哀也。

士大夫

钱穆大师是传统士大夫阶层的典型代表,看他对清朝那种不屑一顾的看法,就知道了。所以说他的学识,研究还是传统士大夫的学识,研究。所以虽然觉得这本书从很多角度都值得看,但毕竟是一家之言,不能无条件全部接受,也不可无条件全部拒绝。

但是,大师知识之渊博,可叹!
