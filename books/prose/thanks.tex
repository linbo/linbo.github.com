\chapter{感恩三年}

\section*{2009~2010}
\addcontentsline{toc}{section}{2009~2010}
\markright{2009~2010}

\begin{quote}
    \begin{center}
纯粹三年的流水帐
\end{center}
\end{quote}

刚毕业的时候,我们班主任饱含深情的教导我们,要记得感恩。这个词我一直记得。想想人类历史存在了几千年,地球那么大,中国人口那么多,居然能有缘分在深圳相遇,除了感恩还能有什么呢?

因为老哥在深圳,所以09年五月份第一次坐飞机,来到了坂田,宅了一个月后,6月1日正式上班。当时一起onboard的一个是南大的,一个是东南大学的。

因为有实习经历,所以很多东西不是那么陌生,也对很多东西有了免疫,除了英语。进来发现这个组是和印度那边的同事合作的,由于我是我们组第一个报道的,为了避免我无所事事,老大和印度那边老大决定让印度人给我做为期2周的培训。依稀记得那两周的电话培训,我一个字都没有听懂,真的一个字都没有听懂哇。每次印度人一问问题,不是Sorry就是OK,那个汗啊。三年来,我用的最多的英语词汇就是Sorry和Ok。

7月份后,五湖四海的兄弟姐妹们,因为共同的一个原因,来到了甲骨文。那年我们组招了6个新人,队伍壮大了许多。新人全部做新项目,大家刚开始都挺努力,都挺Happy。不过话说金融危机能招这么多人,可见甲骨文真的不缺钱。

大概到8,9月份,我们开始划分小组。我和xiaoju去做虚拟化技术那块,其他人做的东西都不同,跟的印度人也不同。于是经常调侃印度人的英语,Fei.Lin老说他Mentor的英语像机关枪,我和Chris他们都感觉挺幸运的。不过Chris的Mentor也不见得那么Nice,一ping就自动下线。我和印度人交流也没有那么顺畅,特别是和印度开发,每次发信基本上石沉大海。大家纷纷感叹交流的困难,但是似乎也没有什么好的办法。只能说万事开头难,现在Team有这样的发展,前面几批人真的花了很多心血的。

9月底正好一个新版本的开始,我刚开始试着写测试用例,接到通知,新人全部去北京ClassOf。所谓ClassOf,就是公司的一些企业文化的培训。于是甲骨文公司2009的那批新人(包括社招),全部被拉到北京很偏远的一个地方,进行了为期一周的培训。那地方据说前北京市长王宝森经常去的地方。

为期一周的培训包括各个方面,现在其实已经没有多大印象了。只记得在哪里吃得好,玩得好,深圳这边的大家新人都熟了,还认识了不少北京同事。也许那正好是秋天,感觉北京的天气还不错。天空很高,空气有些干燥,但是微风拂面,还是很凉爽的。

培训完了后,还跑到大学同学老田家吃喝玩乐了两天。那时正好建国六十周年前期,天天早上有大批大批的学生去天安门排练,而且周围全是警察。当天晚上去天安门广场逛了逛,晚上就封场了。第二天跑去颐和园溜了溜,话说颐和园真大,也非常漂亮。北京,我也算不虚此行了。玩了两天后,纷纷作鸟兽散的深圳新人又在机场集合,一起回到了深圳。

2009年很快就过去了,2010年的这一年,事情正在起着变化。

由于金融危机,2009年上海外企哀鸿遍野,基本上不招人了。记得当时找工作的时候有个小外企,做通信行业的。当时宣讲会很高调的宣称自己有多少现金储备,还顺便讽刺了几家大公司,非常确定的口气说自己今年肯定招新人。后来据说签了三方的同学被这家公司毁约了,可见金融危机肆虐的多么严重。

当时到2010年,情况开始好转。大概3,4月份,以前实习的老大在MSN ping我,问我愿不愿意回去上海。以前老大是MS出来的,现在已经是他的第三个Team了,做的东西是自动化测试工具的开发。当时幼稚的我天真的以为,自己刚来一年,而且老大对我们真的很不错,如果突然离开对公司不是很好,毕竟我们刚刚起步,于是就拒绝了。到现在想想,这个机会真的很不错,也有些后悔。唉,我总是心太软啊。

到了大概五月份,事情开始起了变化。凝聚我们组的Manager要离职了。太不爽了,我因为老大才拒了offer,老大居然先溜了。其他人也纷纷表示同样的愤慨,呵呵。由于我在AD实习过,经历过人事变动,所以没有一般人那么的触动,但是感觉还是非常非常的遗憾。其他人多多少少会有些shock,因为这个比较是老大亲手建立并带大的团队,如今老大走了,很多东西都不一样了。老大一走,Jesse也提出离职。Team已经不是原来的Team了。

因为大家做的事情不一样,所以作出的成绩也不尽相同。一年后,Chris,伟哥和Kingwei他们的成绩最漂亮,特别是DB组的伟哥和Kingwei,发的bug好多。我看了看xiaoju和我的成绩,真是惨不忍睹。

记得实习的时候,老大曾经对我们说过,一定要做team里不可替代的人,这样才不能被淘汰。于是我试着做一些工作,来增进和开发的交流,同时在工作也更加注意提高自己的一些成绩。可惜各种各样的原因,有些东西半途而废,很是遗憾。

可能七八月份吧,classof之前开始的那个版本已经release。HQ的老大大发慈威,决定给一些budget让CDC的测试和开发一起去outing一次。于是xiaoqiu,dan哥就和开发组一起策划、组织了阳朔outing,这是我在甲骨文的最好的一次outing。

outing是9月初的一个周五出发的,坐了估计10个小时的汽车才到阳朔。到了阳朔发觉,那边的山全是石头,基本上没有泥,而且都独立着,好像人为在大地上插了这些巨石,纷纷赞叹大自然真是神奇。当然还有阳朔的空气,真是新鲜啊,感觉呼吸到了老家的空气一样。

由于跟团,流程非常紧凑。第一天早上先去穿银子岩,其实就是在山腹的溶洞了。在里面,不得不感叹大自然的鬼斧神工。里面的溶洞空间很大,气温很低,非常清爽,引得我诗性大发,不过我是神神叨叨的向Chris他们背古诗,哈哈,他们估计我得神经病了。出了银子岩,应该去野人古道吧,不过就是一些山路,几个黑油皮肤的人做幌子,没啥稀奇的。后面又去了遇龙河玩水战,开着小船,玩着水枪,看着阳朔的好山好水,感叹这样时光为何不能多一些。

玩了水战后,一个个湿漉漉的回来了,紧接着下漓江。小时候背了好多课文,比如桂林山水甲天下,漓江的水真*啊。这次来了漓江一看,果然名不虚传,碧绿的水底下悠闲的躺着更碧绿的水草。看得我好像在游康桥:
软泥上的青荇, 
油油的在水底招摇; 
在漓江的柔波里, 
我甘心做一条水草! 

在漓江上游荡了好久,还看了二十元纸币上的风景,尽头是漓江的白马图,据说周恩来看出了9匹。不过我瞧了半天,也没瞧出多少匹。不过即使有更多的人看出更多的马,也比不过这座山看得人多。

第二天的几个景点就不是那么有吸引力了,除了穿山就是爬山,还看了一场少数民族的show,但是没有留下太多回忆。下午我们坐了差不多10小时的车,又回到了深圳。

开发老大Arthur没有参加这次旅行,不过在周一回了一封邮件,贴了一首诗。原来他以前已经去过一次阳朔,还写了一首诗,太欺负人了,我视作这是对我们去的所有人的一种挑衅。正好这几天一些词句老是在我脑海跳动,晚上回去我稍微组织组织,回了一首垃圾的不能再垃圾的诗\footnote{《冷月无声》}给他,也算是迎战了。

钱钟书写的文章《论快乐》里分析道,快乐是短暂的。事实证明,快乐是很短暂的。到了十月份,以前实习的同事又给我推荐了一个职位,跑去上海面了面,拿到了offer。不过同时,也许我平时工作表现尚可,印度那边邀请我去班加罗尔工作一段时间,同时要去的还有Fei.Lin。这弄得我很矛盾,因为Team表面上看歌舞升平,其实内部已经暗流涌动了。不过我还是决定去印度,原因很简单,我没有去过。

于是我狠心的拒了原来同事的好意,决定去班加罗尔。不变的永远是变化,过了几周,Chris也拿到百度offer,决定离开深圳回北京和老婆相聚了,而且xiaoqiu也马上要transfer到其它组去了。这时,Fei.Lin也接受了创业公司的邀请,准备回武汉创业了。看来我要一个人孤零零的去印度了。

后面发生了一些事情,让我有了一些意见,造就了一些不愉快的事情发生。不过后面dan哥也决定去印度,同时去的还有开发的3位同事,让我本来不想去的念头又打消了下去。

就这样,去印度之前,写了小诗送了Chris\footnote{《冷月无声》},Fei.Lin\footnote{《冷月无声》},xiaoqiu\footnote{《冷月无声》}。我也打点行装,收拾心情,准备去印度。


\newpage
\section*{印度行}
\addcontentsline{toc}{section}{印度行}
\markright{印度行}

梁实秋在《雅舍》写道,不论是最是经济的四川雅舍,还是琼楼玉宇,住的久了便有了感情了。这似乎比《陋室铭》俗气多了,但是更富人情味了。与屋如此,何况与人?

曾经写过一篇《游戏盲的悲哀》,感叹我在最为奢华的年龄居然不会玩游戏。不过后来想想,我似乎总是把生活推迟了,在该干啥的时候总在做上一个年龄段的事情,所以总是错过很多风景。Fei.Lin这小子在大学绝对是个玩主,对各种游戏那是头头是道。为了打发无聊的时光,Fei.Lin教我和Chris玩Dota,有事没事,我们就在下班的时候玩两局,就这样我居然学会的Dota,而且发觉其实我还是有一点点游戏细胞的,太浪费了。

还记得和Fei.Lin,Chris一起学车,玩游戏的时光。不过他们走了,快乐时光也随之消失。幸好要去印度了,失落心情被去印度的兴奋心情代替。由于开发组前面已经有人去过印度,所以我们这批就顺利多了,非常容易的搞定签证和酒店。由于dan哥决定的较晚,所以我和开发者的Nicole,Ken,Wilber先打头阵了。

应该是一个周六吧,我们从蛇口出发,直奔HK机场。大概10点多,飞机起飞,目的地班加罗尔。第一次冲出中国,坐这么长时间的飞机。飞机上基本上都是印度人,连晚餐都是印度的,看不懂随便点了一个,那叫一个难吃。

由于时差关系,大概12点多到班加罗尔。第一次出国,填表格,拿行李,搞的手忙脚乱,差点把相机给丢了,幸亏机场人员发现我落了相机。酒店外派有人员在机场,都不知道他们怎么接上头的,就迷迷糊糊的和大家一起坐上汽车回酒店了。

开车小孩不会讲英文,所以基本上没啥交流。机场出来的公路非常不错,以为班加罗尔的基础设施很不错,结果发现大错特错。到后面,汽车在各种小道里七拐八绕,而且路上没有红绿灯,每5分钟道路有个凸起,估计为了防止“70码”吧。就这样颠颠簸簸的把晕乎乎的我们送到酒店。酒店前台的小伙很帅,不过他的英语还是made in India啊。我们连比代划,用made in China的英语搞定手续。回到房间大概已经2点了,洗澡后直扑床被,闷头酣睡。

第二天不知道什么时候起床的,中午点了酒店的午餐,第一次真正的吃印度食物,感觉还可以的。下午在附近转了转,酒店在富人区,周围都是独立的别墅,家家都有汽车。不过基础设施那叫一个烂,感觉整个班加罗尔就是大冲的扩大版,实在和中国没法比。

印度老大帮我们租了一辆车,每天上下班,吃中饭,晚餐,外出活动,都有专车接送。于是我们过上了三包生活,包车,包吃,包住。人生若是如此,岂不快哉!

我们5个人属于5个不同team,我和Nicole,Ken做的是同一个产品,我是QA,所以和他们混的比较多。Ken是个工作狂,不过他们组的人都很爱玩,经常和他们组的人一起吃喝玩乐。Nicole交流能力很强,很快就和她组的同事混熟了,我们也偶尔和她的组一起吃饭。最神的是dan哥了,路上碰到个老头居然都能侃上一会儿。我在印度的同事就2个人,除了和我mentor交流比较顺畅外,和其他的同事就磕磕绊绊,甚是烦恼。

在印度日子过的很简单。每天早上吃完酒店早餐,司机拉我们去公司,中午大概2点左右才吃饭,我和Ken组经常一起跑去中餐馆或者pizza hut吃饭。晚上我们几个就集体行动,出去觅食。班加罗尔的交通也不咋地,道路很窄,而且遍地摩的穿来越去,汽车爬的比蜗牛还慢。不过奇怪的是在车里居然没有太大感觉,估计入乡随俗了吧。

说到吃,不能不说印度的“美食”。听闻有人去印度,带锅,带水,带方便面。在HK机场有人知道我们去印度,还问为什么不带方便面。我自觉吃饭不怎么挑食,应该没有那么恐怖吧。事实证明,不要侮辱大众的智商。第一天我和同事去9楼吃饭,印度同事很多都带饭的,所谓“饭”,就是一些饼和不知道什么弄成的汤,好像根本没有菜的概念。而且他们吃饭真的用手的,后面知道,有些人分左右手,有些人是不分的。我也体味了一把异国风俗,后来我再也没有上过9楼了。

印度人很多是素食主义者,有些是半素食主义者。印度美食很多都是油炸的,连蔬菜也是。印度没有牛肉,因为牛在印度是神,经常看到牛晃晃悠悠的过马路,司机都要停车等候。不过印度人吃鸡肉和海鲜,羊肉,猪肉很少吃。我们在印度只吃过一次猪肉,结果就是那真的是猪肉,咬一口猪身上的味道全在你嘴里。我和Ken经常中午溜出去找吃的,连神人dan哥都不行,到后面都晚上自己做饭吃。好像Nicole比较厉害,经常和她组的人一起吃,非常佩服。如果有人要问在印度英语和印度美食选一个的话,我觉得那相当侮辱我的智商了。

印度等级制度森严,分三六九等。等级越高,受教育程度越高,从事的社会职业越好。其实印度很多下层阶级,受教育程度低,生活在社会最底层。公司外面的马路边上砌着围墙,经常看到有人在墙角小便,整个马路臭烘烘的。白天经常看到很多妇女儿童坐在地上,无所事事。我们司机等级不高,所以只能从事司机这个行业,不过经过交流,发觉他对现状很满意,似乎没有去努力改变现状的意愿。有时候想想,其实中国有些地方要比印度好的多,像印度的女性非常受歧视,特别是下层阶级的家庭女性,而在中国就好多了。

虽然工作日比较简单,但是周末我们几个就没那么安静了,基本上每个周末都跑出去玩。有时候去逛街,整个班加罗尔就一小块地方比较繁华,就几条道路,算是商业街吧。印度的东西大抵比中国便宜的多,像“Lee”的牛仔裤折合人民币大概2/300元吧,而且居然是made in India的。不过印度的披肩比较有名,同事带我们去逛了逛,我们也毫不客气的为印度经济做出了自己应有的贡献。

有时候印度同事还推荐我们去其它地方,有一次去野生动物园,还有一次去一个当地的皇宫,都比较远,坐车坐了好久。印度的旅游景点和中国完全相反,印度旅游景点对自己国家的人非常便宜,但是对外国人非常昂贵,不像我们国家,磨刀霍霍向大家。印度本地的一个皇宫非常富丽堂皇,可惜不能带相机进去,所以只能饱饱眼福而已。不过印度同事都推荐我们去泰姬陵,本来我非常想去,结果弄到后面没去成,现在想想,估计是印度之行最遗憾的事情了。

虽然在印度吃的不咋地,但是对其它各个方面都很满意,特别是工作上。以前那些印度开发都不怎么理你,不过去了印度一趟,熟络多了,IM ping个消息,发个邮件,基本上比较顺畅了。而且经过在印度40多天的奋斗,我负责的自动化基本上可以跑跑了。总之工作上已经往好的方向发展了。后来证明,这趟印度之行的确值得,因为12c的测试我们CDC是主力,我也为此做了小小贡献。

四十多天的很快就过去了,在1月中旬我们准备回国。在印度过了圣诞,新年还有印度的国庆日,终于到了2011,然而动荡还在继续。


\newpage
\section*{我的2011}
\addcontentsline{toc}{section}{我的2011}
\markright{我的2011}

Ken、Wilber和我从印度回到深圳,很多同事都已经回去过年了。Nicole倒是舒服,直接飞回成都了。与班加罗尔相比,深圳的这个时候有些冷清。没过几天,突然听闻整个QA的VP离职了。中国的team都是他一手创立了,没想到他也离开了。离别,似乎成了2011年的主旋律。

过完年,伟哥就在着手transfer的事情。经过有我们组特色的transfer流程之后,五月份,这家伙如愿以偿的去了fussion开发组,一个看上去很美的新组。

全IT都在叫嚣着云计算,我测试的虚拟化模块也升级成私有云模板,正在起着比较重要的作用。可是中国+印度的全部QA才5个人,比例3:2,中国的小组承担着很重的测试任务。私有云模板的三大模块,中国这边的QA就负责了两大部分。这让我很是奇怪,因为据我了解,外企中的小组研发,除非做的是国内的业务,否则很难是嫡系打主力的。不过我们这边的三个人还是很努力的工作,测着那看不到release的EM12cc的虚拟化模块。

已经记不起来什么时候了,我们组第二批同事也去了印度,这次人数众多,浩浩荡荡有4个人。而且Ken和Nicole居然又去了一次印度,看来开发的进度的确很赶,据说他们去了不像第一次那么轻松,直接当苦力了。一回生两回熟,印度人还真一点都不客气。同时我们这边测试的也很辛苦,不管是手工还是自动化,总是问题多多,不过总算能应付的过去。

时间很快,因为release的时间推了又推,到了7月份,室友KaiLong准备离开回杭州了。当年我和KaiLong,zhaojin三个人同一天报道的,如今铁三角终于要分崩离析了。同时差不多时间,Lyna也准备离开深圳回南京了。KaiLong,Lyna我们都是同一批进公司的,如今看看这一批因为金融危机的难兄难弟,如今都找到更好的目的地,飞向更远的地方,为他们高兴的同时,心中甚为惆怅。

终于差不多在9月份,大部分工作都完成了。同事xiaoqiu要回老家结婚,我和伟哥正闲着无事,决定参加xiaoqiu同学的婚礼,同时顺道去厦门玩玩。

我家是三省交界,走着走着说不定就漫游到江西或者福建去了。而且据说我的祖先就是在抗战的时候从福建迁到浙江的,这样看起来我也算半个福建人了。不过我对福建那是一无所知,所以对旅途倒是有着很大的期待。

我们先去了xiaoqiu老家参加她的婚礼。现在的婚礼全中国都差不多了,总感觉很是惋惜,不然应可以观赏到很多很有趣的东西。不过在那边还保留了一个传统,婚礼当天中午,亲朋好友会聚到xiaoqiu家,一起吃卤面。老实说我长这么大,对卤味一点概念都没有,还是第一次吃卤面呢。这让我想起每当听到广东的很多人都没看到过雪,我心理总有一丝得意,语气总有那么一丝自豪,真是井底之蛙!

卤面当真很好吃,第二天xiaoqiu还带我们去吃了卤味早餐,第一次尝很是不错,不过要像那边人那样天天早上吃这个,估计会有点吃不消。其实蛮奇怪为什么福建那么多卤味,估计和四川吃麻辣一样,都是因为地理位置的原因吧。突然又想起阳朔吃的米线了,不知道有没有这样一本书,写各地那么多吃的特色是怎么样发展起来的,一定很有趣。

参加完xiaoqiu同学的婚礼后,我和伟哥就直奔厦门。中午到了厦门后,二话不说,直接上了鼓浪屿。鼓浪屿是厦门对面的一个小岛,很小但是游玩的人特别多,特别多。又想起好像两个地方被水隔着的还不少呢,大陆和台湾,徐闻和海南,厦门和鼓浪屿,浦东与浦西,杭州市区和滨江。这样有个好处,可以提供一个欣赏对面风景的地方。

鼓浪屿据说是很小资的地方,但是我真的没有什么感觉。就是刚上岛的一两条小街还繁华些,但是一点特色也没有,好像阳朔的西街也没啥特色。绕岛转了转,爬了爬岛上风景区,眺了眺台湾,感觉还是夜晚的时候最舒服。悠闲的坐着海边,任凭海风轻抚,看着对面厦门的高楼大厦,灯红酒绿,要是来个闽南歌曲估计更有感觉了。总之,在鼓浪屿和阳朔一样,感觉生活节奏太舒服了,整个人都身心轻松,疲态尽失。

玩了鼓浪屿还在厦大住了一晚,厦大边上就是南普陀,拜佛烧香的人很多。记起来大学选修的一门课有讲中国的寺庙布局,都很模糊了,后来温习了一下。刚进门是天王殿,供奉弥勒佛和四大天王,然后大雄宝殿,供奉三大佛祖,前世的燃灯佛,现世的释迦摩尼,后世的弥勒佛,周边还有十八罗汉。最后就是观音殿,供奉千手观音菩萨。比起中国无事不登三宝殿,印度和其它宗教虔诚的多,每周都需要做礼拜。对待宗教态度大不相同,也挺好玩的。

下午坐着公交去了环岛路,厦门的公交车,等到所有人都上齐后才慢吞吞出发,深圳的公交,人还没上去就直接开动了,难怪深圳生活节奏那么快。环岛路风景真的不错,早上跑跑步一定相有着“海风轻轻吹,海浪轻轻摇”的惬意。晚上去了久负盛名的中山路美食街,和北京的王府井、上海的南京路一样,这些商业街都大同小异。黄则和人太多没去成,其它小吃没有太多感觉。1980烧肉粽感觉和老妈做的没啥特别;麻糍太小,混囵一吞,已成腹中餐;烧仙草就是小时候家里经常做的神仙豆腐,不过我们是用来做菜的。不过中山路有好多漂亮的女孩子,估计一方水土养一方人,厦门女孩的腿普遍细长白皙,很美,很是好看。

厦门行就这样匆匆结束了,又回到忙忙碌碌的深圳,那时也快十月份了。


\newpage
\section*{离别}
\addcontentsline{toc}{section}{离别}
\markright{离别}

再不写就忘记了

已经记不清楚EM12c是什么时候release的了,大概在10月份吧。作为一种惯例,老大们为了犒劳我们,于是又有了一次outing,居然还是去厦门。就这样,不到一个月,我第二次踏上厦门的土地。

由于跟团,行程紧凑,被拉去购物的时间占了一大半。鼓浪屿就挤上去几小时,走马观花的又回来了。乘风破浪的和“三民主义”打了个招呼,风尘仆仆的路过环岛路,半途而废的爬了半山的南普陀,九曲十八弯的去了永定土楼。对我来说,唯一的亮点就是土楼了,以前还没有见过这种建筑,感觉蛮新奇的,可惜现在的土楼基本上已经人去楼空了,很难有以前那种几世同堂的壮观了。

铁打的营盘流水的兵,对任何一所学校,任何一支军队,任何一家公司,任何一个team来说,都是如此。时间永是流驶,街市依旧太平,有限的几个人员流动,对team来说,都不算什么,我们team也是如此。大概在11月,接替Lyna的Abby入职,看上去我们小组有日趋稳定的趋势。然而不变的永远只有变化,很短时间内,Terry决定转到其它组去,这就造成中国team这边招不到人,印度那边大肆扩张,本来可以势均力敌的平衡被彻底打破,我们小组还是逃脱不了外企小组的命运。

时间很快,马上到了2012,但是新人来,旧人走的状况还是没有改变。Fussion Middleware连续三员大将转组,基本上整个组经历了4年,推倒重建了。当然同时陆陆续续有新人进来,只是在O记,属于我们那一批的表演基本上已经结束,帷幕已经降落。

从2010年,就断断续续有在面试,不过2011年的面试,基本上只是一种体验的心态。原因很简单,当时做测试的空间并不大,在里面多多少少如同井底之蛙,还是需要出去看看同行的情况。面了几次之后,发现由于行业的不同(我主要面的是国内的IT互联网公司),至少在技术上,自己还是和这些公司有一些差距的。很多和我一样,一毕业就进入O记的人来说,待了2,3年,难免心里发虚。一方面某些小组提升的空间并不是很大,另一方面太不了解外面的世界,可能导致浮躁不已,可能导致毫无信心。总之,多去了解了解外面是世界,还是很有必要的。这样既不会盲目崇拜,也不会目空一切。

到了2012年,我基本上已经没有参加任何面试了,不过同时在更有目的的着手准备。4月份,我在小组的最后一次活动,参加校园intern招聘。我们小组一大半的人,浩浩荡荡去了中大进行校园招聘。还是第一次参加校园招聘,以前看着台上那些师兄师姐,侃侃而谈,心里想着哪天自己也可以站在上面,不过现在看来,估计还得等几年吧。一直觉得自己还是刚毕业的学生,但是当参加校园招聘时,发觉自己已经是老人了,人家都喊你师兄了,真是师兄无情,师姐无义啊。多亏了Cindy她们的安排,整个校园招聘还是很顺利的,这次team活动,还是很愉快的。最后收获颇丰,当然最后这些实习生到底怎么样了,纵有心却无力,我就爱莫能助了。

五月份,quanquan一通电话,怂恿我去他们公司看看,于是我真的只是去看看。看看的意思,就是和老大聊了不到10分钟,不涉及任何具体的技术问题,只是大家大概有了个了解。看完之后,和几位同事/前同事交流了一下,大家都支持我离开。当时在O记,已经有一条看上去可能还不错路,假若一切安好,走着也还过得去。可惜生不逢时,在这种年代,那条路太不具有中国特色了。对我个人来说(对别人不一定适合),要想在现今社会过得有意思一点,风险小一点,我唯一能做的,就是武装自己,壮大自己,让自己以后能走各种不同的道路。

于是就这样,我决定离开,换个环境,换一条道路走走。估计一些O记的同事对我的选择犯嘀咕,但是原因真的很简单:为了以后老了,能多吹吹牛,我现在只能努力的多找点素材。正如CCTV5那厮(贺炜)所说的:当观众朋友老去以后,在壁炉旁边抱着自己的孙子,给他讲世界杯故事的时候,一定不会忘记2010南非世界杯这场赛事。当然越简单的事情可能越容易复杂化。

从2009年6月1号入职,到2012年6月1号离职,整整三年,一秒不差。三年有很多欢乐,有一些迷惘,有几刻不爽。最最重要的,是遇到很多人,交了好多好朋友,不是最好的时光里有你们在,而是你们在,我才有了最好的时光。亲,你们懂的!

有人说,离别是为了再次相遇,有人道,离别是情已了、缘已尽,有人言,离别是为了一个新的开始。纵有千种缘由,多情自古伤离别。“桃花流水三千尺,不及王伦送我情”的踏歌而别;“劝君更尽一杯酒,西出阳关无故人”的悲凉饯别;“莫愁前路无知己,天下谁人不识君”的壮志之别;“风萧萧兮易水寒,壮士一去兮不复返”的生死离别。人非草木,怎敢无情,即已多情,不言离别。

为了纪念O记三年,涂几笔,描几画,以此来感恩。
