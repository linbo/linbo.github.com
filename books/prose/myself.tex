\chapter{我的自白}
\section*{一}
\addcontentsline{toc}{section}{一}

吾有名有姓,无字无号。于一九八四年六月十九作客地球,甚留恋,不忍回去,至今有二十余载。问归期,答曰:遥遥无期!

年方十,世人竞相奔走相告,遥说世间有象牙塔一座,进之,则如同鲤鱼跳龙门,便可咸鱼翻身。读书人趋之若骛,吾奇之,遂加入其伍。途间披荆斩棘,呕心沥血,披星戴月。今想之,唏嘘不已,百感交集,感慨万分也!

年二十,了心愿。然象牙塔外表华丽,其实空虚。终日游手好闲,无所事事。于塔之余晖下,躲此之中,不知世间有春夏秋冬,更何论今夕是何年。

然江水东逝,非我等所能阻也,时光飞奔,非我等所能留也。终有一日,将离塔,随意飘荡,任意东西,游荡四处,飘零八方。念之,竟慌而不知所措。故发贴,留函,望世间高手能指点迷津,知无不言,言无不尽。如能得三言四拍,感激不尽!

\newpage
\section*{二}
\addcontentsline{toc}{section}{二}

十月,窥哲学、玩诗词,思虑甚重、用力过猛。于是乎,常观月明星稀,思月宫之寒,穷万物之致,而笑世间之薄陋,竟嬉戏而不能自拔,至夜不能寐。久之,心神俱疲,心力交瘁。

入十一月,陪妹游东方之珠,妹惊曰:哥瘦。友见叹:汝亦瘦焉。二旬未见,哥诧异:弟似又瘦乎。遂求医,访友,共娱乐,其乐融融。虽体态仍若细柳扶风,然心态渐宽,不敢心较比干多一窍也。

吾非陶公谢公,不念世间众人,不惦凡间诸事;亦非范公文公,思其君忧其民,一片丹心照汗青。吾胸无大志,仍希翼走之足迹,乃一道自己之风景。

年年岁岁,如一江春水,永世流逝,不可留也。世间万物,若阴晴圆缺之月,起落峰谷之波,长久异也。吾岂能故步自封?不望今日之吾异于昨日之吾,但盼今年之吾异于旧年之吾。
