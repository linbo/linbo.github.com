\chapter{胡思乱想}

\section*{你们该沉默沉默了-我看“文笔门”}
\addcontentsline{toc}{section}{你们该沉默沉默了-我看“文笔门”}
\markright{你们该沉默沉默了-我看“文笔门”}

自从“水门”事件后,中国人就喜欢对各种各样的事件走个后门。比如“老虎门”,“铜须门”,“艳照门”等等。我是中国人,不用入乡随俗,自然而然的就在韩寒事件走个后门,我叫它“文笔门”。

还有一位主角值得介绍一下,他叫陈丹青。一说起他,我的思绪就回到了中学。我们高中有位很有才华的美术老师,叫郑建中。有节课他拿了一本杂志,读了一篇文章,是陈丹青写的,顺便老师还介绍了一下陈丹青。那篇文章是陈丹青回忆他的老师颜文良的,文章写的很好,但是郑老师的赏析更好。同样是学美术的,文学修养比我见到的语文老师都好。以至于我一直很羡慕学画画的这些人,感觉这些人才叫多才多艺。

我看了看韩寒的这篇文章,特别是后面评论的摘录,感触良多。

我是学理科的,所以对文笔这个抽象概念还真的不是很清楚,不过我倒同意韩寒的观点。我十分赞同这个说法:我不同意你的观点,但是我誓死捍卫你说话的权利。看看那些评论者一副道德圣人的样子,高高在上,指手划脚,乱扣帽子;写的评论就像八股文,内容空洞,毫无逻辑,说的都是一些无关痛痒的大话,有的更像泼妇骂街一样的没头没脑,真让人可气又可笑。

其实文学作品是有好多部分组成的,文笔只是其中的一个很小的部分。我认为文学大师之所以称之为大师,是他的文学作品总体水平较高,或者说他文学作品中的某些部分到达一个非常非常高的高度。但是大师的作品不可能每一部分都是完美无缺的,否则那不叫作品了,叫圣书。大师也不叫大师了,叫圣人。而不管是圣人还是圣书,都是不存在这个世界上的。所以大师作品的某一部分没有达到一定水平,是很正常的,这不会抹掉他作为大师的资格。

韩寒和陈丹青从自己的角度出发,认为一些大师作品的某些部分水平不高,碍大家什么事了。有些人如丧考妣,破口大骂,和晚清的那些遗老遗少没了皇帝一个样。而且作为一个评论者,没有一些证据,在那里哗众取宠,胡说八道,完全一副没头没脑的随从跟班的模样。一看到这种东西,我好像大话西游里的孙悟空,听到唐僧的废话,感觉好像好多好多苍蝇在嗡嗡乱叫。真恨不得做一些动作,让这个世界清净一点。

中国的道德高人太多了,说大话的人也太多了,这些家伙,你们该沉默沉默了。



\newpage
\section*{七夕的怪想法}
\addcontentsline{toc}{section}{七夕的怪想法}
\markright{七夕的怪想法}

一直不知道七夕是什么节日,后来道听途说,外加自己网上搜了一下,才发觉,原来七夕就是牛郎织女鹊桥相会日。搞了半天,中国情人节就是七夕加个洋标签,made in china啊!现在人估计都不大知道,因为抬头望望天,根本不知道银河流到哪里去了,不过倒是知道中国情人节。中国五千年的历史终于在这个时候起作用了。遇到一个中国没有的什么节日,在五千年的历史长河中,总能找到一点蛛丝马迹,于是结合中国国情,就变成了中国制造。幸亏节日不能当作商品,否则七夕出口到国外,被老外一检查,来个历史含量超标,我泱泱大国,颜面何在?

对于在中国的单身汉来说,可有得受了。在茕茕孑立,形影相吊的一年中,过两次不属于自己的情人节,外加一次光棍节。看来,孤独得人是可耻的,也不是没有道理。

不过,关于七夕的怪想法和上面的文字可是一点关系都没有。我一直在关注这样一个问题,那就是经常有什么新的节日出现。照这样下去,一年365天,总有到每天都是过节的共产主义。到那时,我们都不用日历了,每人每天出门必带的是节日大全,就像今天手机里的万年历一样。两情人一见面,

男曰:前天是什么日子?(相当于问前天几号)

女(翻翻节日大全)答:植树节。(3月12号)

男曰:今天是警察节吧,那明天不就是国际消费日吗?时间过得真快啊!我感觉情人节就像昨天刚过的。(国际警察日[03/14],国际消费日[03/15])

女答:不要紧,再过几个月,不是还有中国情人节吗?
