\chapter{波心荡}
\thispagestyle{empty}

\renewcommand{\poemtoc}{section}
\poemtitle{波心荡}
\settowidth{\versewidth}{世人皆谓蜀道难}
\begin{verse}[\versewidth]
世人皆谓蜀道难\\
我笑何人曾试攀?\\
待与天公千杯醉\\
一梦踏尽蜀汉山
\end{verse}

\renewcommand{\poemtoc}{section}
\poemtitle{夜饮}
\settowidth{\versewidth}{花间一壶酒}
\begin{verse}[\versewidth]
花间一壶酒\\
水中一弯月\\
酒伴人销愁\\
月伴人空瘦
\end{verse}

\renewcommand{\poemtoc}{section}
\poemtitle{温柔}
\settowidth{\versewidth}{最是低头一温柔}
\begin{verse}[\versewidth]
最是低头一温柔\\
胜似春风拂心头\\
欲语牛郎织女夜\\
却话一声道珍重
\end{verse}

\renewcommand{\poemtoc}{section}
\poemtitle{群英会}
\settowidth{\versewidth}{群英邀我共赴会}
\begin{verse}[\versewidth]
群英邀我共赴会\\
天公怎敢不作美\\
纵使漫天飞雪舞\\
寒江独钓岂一人?
\end{verse}

\renewcommand{\poemtoc}{section}
\poemtitle{独英会}
\settowidth{\versewidth}{群英邀我共赴会}
\begin{verse}[\versewidth]
我邀群英共赴会\\
天公竟也不作美\\
漫天飞雪追逐戏\\
寒江独钓仍一人!
\end{verse}

\renewcommand{\poemtoc}{section}
\poemtitle{战书}
\settowidth{\versewidth}{秋高气爽 QA射雕}
\begin{verse}[\versewidth]

\emph{\scriptsize{小组技痒,向Dev下挑战书。于是老夫捉刀,豪言片语}}

秋高气爽\ \ QA射雕\\
铁骑所踏\ \ 无不臣服\\
甲骨诸国\ \ 兵败千里\\
今统精兵良将\ \ bug万千\\
欲与足下会猎于科北\\
切磋技艺\ \ 逐鹿中原\\
盼足下养精蓄锐\\
免遭屠戮 
\end{verse}

\renewcommand{\poemtoc}{section}
\poemtitle{标点符号}
\settowidth{\versewidth}{我满怀问号走进了你}
\begin{verse}[\versewidth]
我满怀问号走进了你\\
得到的\\
却全是句号\\
没有分号\\
也没有叹号\\
\end{verse}


\renewcommand{\poemtoc}{section}
\poemtitle{天空的眼泪}
\settowidth{\versewidth}{秋高气爽 QA射雕}
\begin{verse}[\versewidth]
地上的水珠\\
向往天空的自由\\
遂化作片片白云\\
悠悠的飘浮\\
~\\
飘渺的云儿啊\\
那般地爱着大地\\
终究化作天空的眼泪\\
默默的滴落
\end{verse}

\renewcommand{\poemtoc}{section}
\poemtitle{忆故乡}
\settowidth{\versewidth}{溪畔弯柳青青舒}
\begin{verse}[\versewidth]
故乡忆\\
最忆是初春\\
~\\
溪畔弯柳青青舒\\
一波绿水卷春来\\
能不忆故乡?
\end{verse}

\renewcommand{\poemtoc}{section}
\poemtitle{鹏城四年}
\settowidth{\versewidth}{四年偏安南隅中}
\begin{verse}[\versewidth]

四年偏安南隅中\\
竟也一事仍无成\\
回望少年风流梦\\
一片热血洒刀枪
\end{verse}

\renewcommand{\poemtoc}{section}
\poemtitle{书言}
\settowidth{\versewidth}{四年偏安南隅中}
\begin{verse}[\versewidth]
清茶一杯明月照\\
枯灯半盏鸡声闻\\
岂敢年华悄然逝\\
闲坐白头梦少年\\
\end{verse}

\renewcommand{\poemtoc}{section}
\poemtitle{唱和}
\settowidth{\versewidth}{弦断声可续}
\begin{verse}[\versewidth]
弦断声可续?\\
古调谁人听\\
甚好虫鸟声\\
仍唱天地间
\end{verse}

\renewcommand{\poemtoc}{section}
\poemtitle{伊声}
\settowidth{\versewidth}{浓睡愈憔悴}
\begin{verse}[\versewidth]
浓睡愈憔悴\\
懒起倦梳头\\
思君赏花泪\\
奋起画峨眉
\end{verse}

\renewcommand{\poemtoc}{section}
\poemtitle{无题一}
\settowidth{\versewidth}{风仍劲, 雪正紧}
\begin{verse}[\versewidth]
风仍劲, 雪正紧\\
铁骑挺枪驱敌贼\\
满天红花坠\\
~\\
酒更烈,鼓最鸣\\
将军扬刀斩单于\\
千里不留人
\end{verse}

\renewcommand{\poemtoc}{section}
\poemtitle{无题二}
\settowidth{\versewidth}{风仍劲, 雪正紧}
\begin{verse}[\versewidth]
冷风浓,残花落\\
满地嫩绿迎春来\\
何人送春去\\
~\\
将军饮,伊人寐\\
扬鞭策马弓箭行\\
谁见梦中泪
\end{verse}

\renewcommand{\poemtoc}{section}
\poemtitle{无题三}
\settowidth{\versewidth}{风仍劲, 雪正紧}
\begin{verse}[\versewidth]
雨夜冷,街角寂\\
雨珠落地雨花溅\\
雨中谁独行?\\
~\\
热泪涌,深巷幽\\
泪珠成串泪如瀑\\
泪中雨独行
\end{verse}
